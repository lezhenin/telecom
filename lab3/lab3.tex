\documentclass[a4paper,14pt]{extarticle}

\usepackage[top=2.5cm, bottom=2.5cm, left=2.5cm, right=2.5cm]{geometry}
\usepackage[utf8]{inputenc}
\usepackage[russian]{babel}
\usepackage{graphicx}
\usepackage{caption}
\usepackage{subcaption}
\usepackage{chngcntr}
\usepackage{amsmath}
\usepackage{amsfonts}
\usepackage{pgfplots}
\usepackage{pgfplotstable}
\usepgfplotslibrary{fillbetween}
\usepackage{float}
\usepackage{lipsum}% http://ctan.org/pkg/lipsum
\usepackage{multicol}% http://ctan.org/pkg/multicol
\usepackage{hhline}
\usepackage{tabularx}
\usepackage{tikz,xcolor}
\usepackage{tkz-graph}
\usepackage{float}
\usepackage{mathtools}
\usepackage{todonotes}
\usepackage{listings}
\usepackage[makeroom]{cancel}

\usetikzlibrary{arrows, petri, topaths}

\counterwithin{figure}{section}
\counterwithin{equation}{section}
\counterwithin{table}{section}

\DeclareMathOperator*{\argmin}{arg\,min}
\DeclareMathOperator*{\argmax}{arg\,max}
\DeclareMathOperator{\sinc}{sinc}

\definecolor{mygreen}{RGB}{28,172,0} % color values Red, Green, Blue
\definecolor{mylilas}{RGB}{170,55,241}

\lstset{language=Matlab,%
  %  basicstyle=\color{red},
    breaklines=true,%
    morekeywords={matlab2tikz,ylim,xlim,square,ones,double},
    keywordstyle=\color{blue},%
    morekeywords=[2]{1}, keywordstyle=[2]{\color{black}},
    identifierstyle=\color{black},%
    stringstyle=\color{mylilas},
    commentstyle=\color{mygreen},%
    showstringspaces=false,%without this there will be a symbol in the places where there is a space
    numbers=left,%
    numberstyle={ \color{black}},% size of the numbers
    numbersep=15pt, % this defines how far the numbers are from the text
    emph=[1]{for,end,break,switch,case,otherwise},emphstyle=[1]\color{red}, %some words to emphasise
    %emph=[2]{word1,word2}, emphstyle=[2]{style},    
}


\begin{document}
\begin{titlepage}
\centering 
{\bfseries Санкт-Петербургский Политехнический Университет} \\
Институт компьютерных наук и технологий \\
Кафедра компьютерных систем и программных технологий \\
\vspace{5cm}
{\centering \textbf{Отчёт по лабораторной работе №3} \\ 
\vspace{0.15cm}
\textbf{Дисциплина}: Телекоммуникационные технологии \\
\vspace{0.15cm}
\textbf{Тема}: Линейная фильтрация.} \\
\vspace{4cm}
\hfill {\bfseries Работу выполнил студент}  \\
\hfill гр. 33501/4 Леженин Ю.И. \\
\hfill {\bfseries Преподаватель}  \\
\hfill Богач Н.В.
\vfill
Санкт-Петербург \\
{\large \today\par}
\end{titlepage}

\section{Цель работы.}

Изучить воздействие ФНЧ на тестовый сигнал с шумом.

\section{Постановка задачи.} 

Cгенерировать гармонический сигнал с шумом
и синтезировать ФНЧ. Получить сигнал во временной и частотной
областях до и после фильтрации. Сделать выводы о воздействии
ФНЧ на спектр сигнала.

\section{Ход работы.}


\section{Выводы.}

%Преобразование Фурье позволяет представить сигнал в базисе гармонических колебаний разной частоты. Это значительно облегчает обработку и синтез сигналов.
%
%Важнейшими свойствами ПФ являются теорема о свертывании сигналов и теорема о перемножении сигналов: произведение функций - это свертка их образов, свертка функций - это произведение их образов.
%
%Спектр периодического сигнала дискретный, а дискретного - периодический. Если сигнал конечный, то при выполнении преобразования его образ сворачивается с функцией $\sinc(\pi f)$. 
%
%Корреляция дает возможность определить степень схожести двух сигналов, расчет с учетом смещения позволяет определить и компенсировать временные задержки. Данный инструмент часто применяется для поиска известной последовательности во входном сигнале.

\section{Приложение.}

\lstinputlisting[frame=single, caption=Программа для генерации и фильтрации сигналов.]{lab3_script.m}


\end{document}